\documentclass{article}[12pt]
\usepackage[margin=1in]{geometry}
\title{Computational Chemistry \\ \small 3D Modeling and Bond Geometry}
\author{Rutuj Gavankar}
\date{11 December 2017}
\begin{document}
	\maketitle
	\section{Project Summary}
	This project uses an open-source file format called `.xyz', which is essentially a text file with the names and coordinates of atoms of a compound. 
	
	The program reads this text file, coverts it into a multidimensional array, imports a CVS file that contains the periodic table and converts it into an array. It calculates the bond lengths, bond angles, molecular mass, and plots it in 3D space. It also generates a unique colour for each of the atoms.
	
	\section{Use Case Analysis}	
	This project is for people who would like to visualize chemical bonds. Covalent compounds have specific geometries depending on the number of lone pairs and the bonds. The bond length, bond angle, both depend on the electrostatic forces of attraction and repulsion. 
	
	XYZ is an open source file format. Downloading a xyz file of the desired compound and loading it in the program, the molecule can be graphed in 3d space and the molecular mass, atomic radii, bond length and bond angles can be calculated.
	
	\section{Data Design}
	am using arrays for everything. The xyz text file is stored in a 2D array. The CVS file is stored in a 2D array as well. The program is divided into functions. Each function does one job and returns a value. There are also functions that show the values of other functions.
	
	I did not choose object oriented programming because it had no significant advantage over this approach. Infact, it would be twice as much tedious. All functions more or less depend on matrix manipulation. 
	
	There is also a function to save the output as a txt file
	
	\section{UI Design}
	I am using a Jupyter Notebook. The module I am using, VPython, doesn’t work outside Jupyter. 
	The first cell is just the code for people to view
	The second cell executes the 3d plot
	The third cell executes the math outputs
	The fourth cell asks for a save file prompt
	
	\section{Algorithm}
	The XYZ file is imported into python using file functions. It is stored in a multidimensional array (2x2).The same is done to a CVS file with the periodic table. 
	\begin{enumerate}
		\item
		Atomic Mass is calculated by adding up all the masses of individual atoms. The masses of atoms are found by comparing the periodic table cvs and the xyz file. 
		\item 
		Covalent Radius of an atom is displayed by using a similar method
		\item 
		Bond length is calculated by calculating the distance between all the atoms, and then comparing each of them for finding the ones that bond.
		\item 
		Bond Angles are found by converting each of the coordinates to a position vector. The difference between 3 position vectors gives 2 vectors with 1 common point. Angle between these vectors is arc cosine of the dot product of them divided by product of their lengths. It is then converted to degrees.
	\end{enumerate}
	\section{Acknowledgment}

	Modules:
	\begin{itemize}
		\item SciPy : https://www.scipy.org/
		\item Visual Python (VPython): http://www.vpython.org/
		\item NumPy : http://www.numpy.org/
		\item Jupyter : http://jupyter.org/
	\end{itemize}
	Files
	\begin{itemize}
		\item periodicTable.cvs : https://github.com/andrejewski/periodic-table/blob/master/data.csv
		\item xyz files : https://github.com/tmpchem/computational\_chemistry/tree/master/geom/xyz
	\end{itemize}
	Other Resources:
\begin{itemize}
	\item 	https://cs.iupui.edu/$\sim$aharris/230/
	\item	Stewart, James. Calculus. Cengage Learning, 2016.
\end{itemize}
	
	
	
	
	 
	

	
	
\end{document}